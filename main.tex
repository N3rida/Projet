\documentclass{article}
\usepackage[french]{babel}
\usepackage[T1]{fontenc}

\usepackage{graphicx} % Required for inserting images
\usepackage{amsmath}
\usepackage{amsfonts}
\usepackage{tikz}

\title{Système de Lorenz}
\author{Salomé COUDIERE, Louka OUKALA, Adrien CHABERT}
\date{Semestre 6}

\newcommand*\colv[1]{
\left(\begin{array}{c}
    #1
\end{array}\right)
}

\newcommand{\R}{\mathbb{R}}

\newcommand{\deriv}[3][ ]{
    \ensuremath{\frac{\mathrm{d}^{#1}#2}{\mathrm{d}^{#1} #3}}
}

\newcommand{\cad}{c'est-\`a-dire }

\begin{document}

\begin{equation}
    \label{Lorenz}
    \left\{
    \begin{array}{rl}
        x' &=\sigma(y-x) \\
        y' &=\rho x -y - xz\\
        z' &=xy - \beta z
    \end{array}
    \right.
    \begin{array}{r}
        (L_1)\\
        (L_2)\\
        (L_3)
    \end{array}
\end{equation}
On peut récrire ce systeme de la manière suivante.
\begin{equation}
\deriv{\Vec{u}}{t} = \Gamma(\Vec{u}), \quad
\Gamma : \Vec{u} = \colv{x\\y\\z} \in \R^3 \mapsto \colv{\sigma(y-x) \\ \rho x-y-xz \\ xy-\beta z}
\end{equation}
Remarque: $\Gamma$ est polynomiale, donc $\Gamma$ est de classe $C^\infty$ en particulier elle est $C^1$.

\underline{\textbf{Proposition:}} Les solutions du systeme de Lorenz sont globales sur $\R_+$\\

Demonstration:
$\Gamma$ est $C^1$ donc elle est localement Lipschitzienne, de plus elle de depend pas directement du temps. D'après le théoreme de Cauchy-Lipschitz, l'équation \eqref{Lorenz} admet une unique solution maximale de classe $C^1$ que l'on notera $(I,(x,y,z))$ avec $I \subset \R_+$ avec $I = ]0,T[,\ T \in ]0,+\infty]$. Montrons que $(I,(x,y,z))$ est globale. Dans \eqref{Lorenz} on s'intéresse à la quantité: $x(L_1) + y(L_2) + z(L_3)$
\[
    xx'+yy'+zz' = \sigma yx - \sigma x^2 + \rho xy - y^2 -xyz + xyz - \beta z^2\\
\]  
Posons $\mathcal{N}: (t) \in \R^3 \mapsto x(t)^2 + y(t)^2 + z(t)^2$ ($\mathcal{N}$ est la norme euclidienne au carré)

\begin{align*}
    \Rightarrow \frac{1}{2}\deriv{ }{t}\mathcal{N}(t) & =(\sigma + \rho)xy -\sigma x^2 - y^2 - \beta z^2\\
    & \le (\sigma + \rho)xy +\min (1,\sigma,\beta)\mathcal{N}(t)\\
    & \le (\frac{\sigma+\rho}{2})(x^2 + y^2) +\min (1,\sigma,\beta) \mathcal{N}(t) &&\mathit{(Young)}\footnotemark \\
    & \le (\frac{\sigma+\rho}{2})(x^2 + y^2 + z^2) +\min (1,\sigma,\beta) \mathcal{N}(t)\\
    & \le \bigg[\frac{\sigma + \rho}{2} + \min (1,\sigma,\beta) \bigg] \mathcal{N}(t)
\end{align*}
\footnotetext{$\forall p,q \in \mathbb{N}\: \text{tels que} \frac{1}{p}+\frac{1}{q}=1 \Rightarrow \forall a,b \in \R \: ab \le \frac{a^p}{p}+\frac{b^q}{q}$}

Posons $ \eta = \sigma + \rho - 2 \min (1,\sigma,\beta))$. On a alors: 
\[
    \forall t \in \R_+, \  \deriv{}{t}\mathcal{N}(t) \le \eta\  \mathcal{N}(t)
\]
D'après le lemme de Grönwall il vient:
\[
    \forall t \in \R_+,\  \mathcal{N}(t) \le \mathcal{N}_0 e^{\eta t},\  \textrm{avec } \mathcal{N}_0 = \mathcal{N}(0)
\]
Donc la norme du vecteur solution n'explose pas en temps fini.\\

%solution C inf
\underline{\textbf{Proposition:}} Les solution de \eqref{Lorenz} sont $C^\infty$\\

Demonstration:
 Par définition de \eqref{Lorenz} on a que $(x',y',z') = \Gamma(x,y,z)$, or par composition $\Gamma(x,y,z)$ est $C^1$ donc $(x',y',z')$ l'est aussi ainsi $(x,y,z)$ est $C^2$.De la même manière on obtient par récurence immédiate que $(x,y,z)$ est $C^\infty$\\

%pts stationnaires
On cherche maintenant les points stationnaire de \eqref{Lorenz}.\\
On remarque que $(0,0,0)$ est un point stationnaire, en effet $\Gamma(0,0,0) = 0_{\R^3} \equiv 0$ donc ($\R$,0) est une solution de l'equation differentielle.\\
On resout alors $\Gamma(x,y,z)=0$ en supposant que $(x,y,z) \neq 0$, il vient:
\[
\left\{\begin{array}{rl} %O of Gamma
     \sigma(y-x)&=0  \\
     \rho x -y -xz&=0\\
     xy - \beta z&=0
\end{array}\right.
\begin{array}{c} %Num eq
    (L_1)\\
    (L_2)\\
    (L_3)
\end{array}
\]
de $(L_1)$ on obtient que $x=y$. Dans $(L_2)$ on remplace $y$ par $x$, il vient alors:
\[
    (L_2) \Rightarrow \rho x - x - xz = 0 \Rightarrow x (\rho -1 -z ) = 0 \Rightarrow z = \rho -1
\]
De m\^eme dans $(L_3)$
\[
    (L_3) \Rightarrow x^2 - \beta z = 0 \Rightarrow z = \pm \sqrt{\beta z}
\]
De ces trois équation on obtient que:\[
    \Gamma(x,y,z)=0 \Rightarrow (x,y,z) = (\pm \sqrt{ \beta (\rho -1)} ,\pm \sqrt{\beta (\rho -1)}, \rho -1)   
\]
On verifie aisaiment que cette relation est une \'equivalance, en ramplacant les valeurs obtenue de $x$,$y$ et $z$ dans $\Gamma(x,y,z)$\\
On se propose d'étudier la stabilité des points stationnaires. Pour cela on s'intéresse à la linéarisé de \eqref{Lorenz}, donné par:
\[
\colv{x'\\y'\\z'} = \mathcal{D}_{\Gamma}(x_s,y_s,z_s)\colv{x\\y\\z}    
\]
avec $(x_s,y_s,z_s)$ les coordonnées des points stationnaire, $\mathcal{D}_{\Gamma}(x,y,z)$ la differentielle de $\Gamma$ donné par:
\[
\mathcal{D}_{\Gamma}(x,y,z) =
\begin{pmatrix}
    \sigma & \sigma & 0 \\ \rho - z & -1 & -x \\ y & x & - \beta
\end{pmatrix}
\]
On étudie premièrement l'équilibre autour de $0_{\R^3}$:\\
L'équation ainsi obtnue est:
\begin{equation}
    \colv{x'\\y'\\z'} =
    \begin{pmatrix}
        - \sigma & \sigma & 0 \\ \rho & -1 & 0 \\ 0 & 0 & - \beta
    \end{pmatrix}
    \colv{x \\ y \\ z}
\end{equation}
Autrement dit on obtient:
\begin{equation}
    \left\{\begin{array}{lr}
        x' = \sigma (y-x) \\
        y' = \rho x - y \\
        z' = \beta z
    \end{array}\right.
\end{equation}



\end{document}
