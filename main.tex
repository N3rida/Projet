\documentclass{article}
\usepackage[french]{babel}
\usepackage[T1]{fontenc}

\usepackage{graphicx} % Required for inserting images
\usepackage{amsmath}
\usepackage{amsfonts}
\usepackage{tikz}

\title{Système de Lorenz}
\date{Semestre 6}

\newcommand*\colv[1]{
\left(\begin{array}{c}
    #1
\end{array}\right)
}

\newcommand{\R}{\mathbb{R}}

\newcommand{\deriv}[3][ ]{
    \ensuremath{\frac{\mathrm{d}^{#1}#2}{\mathrm{d}^{#1} #3}}
}
\newcommand{\id}[1][]{\ensuremath{\mathrm{Id}_{#1}}}

\newcommand{\cad}{c'est-\`a-dire }

\begin{document}

\begin{equation}
    \label{Lorenz}
    \left\{
    \begin{array}{rl}
        x' &=\sigma(y-x) \\
        y' &=\rho x -y - xz\\
        z' &=xy - \beta z
    \end{array}
    \right.
    \begin{array}{r}
        (L_1)\\
        (L_2)\\
        (L_3)
    \end{array}
\end{equation}
On peut récrire ce systeme de la manière suivante.
\begin{equation}
\deriv{\Vec{u}}{t} = \Gamma(\Vec{u}), \quad
\Gamma : \Vec{u} = \colv{x\\y\\z} \in \R^3 \mapsto \colv{\sigma(y-x) \\ \rho x-y-xz \\ xy-\beta z}
\end{equation}
Remarque: $\Gamma$ est polynomiale, donc $\Gamma$ est de classe $C^\infty$ en particulier elle est $C^1$.

\underline{Proposition:} Les solutions du systeme de Lorenz sont globales sur $\R_+$

Demonstration:
$\Gamma$ est $C^1$ donc elle est localement Lipschitzienne, de plus elle de depend pas directement du temps. D'après le théoreme de Cauchy-Lipschitz, l'équation \eqref{Lorenz} admet une unique solution maximale de classe $C^1$ que l'on notera $(I,(x,y,z))$ avec $I \subset \R_+$ avec $I = ]0,T[,\ T \in ]0,+\infty]$. Montrons que $(I,(x,y,z))$ est globale. Dans \eqref{Lorenz} on s'intéresse à la somme de, $x$ fois première ligne avec $y$ fois la seconde ligne et $z$ fois la troisième ligne.
\[
    xx'+yy'+zz' = \sigma yx - \sigma x^2 + \rho xy - y^2 -xyz + xyz - \beta z^2\\
\]  
Posons $\mathcal{N}: (t) \in \R^3 \mapsto x(t)^2 + y(t)^2 + z(t)^2$ ($\mathcal{N}$ est la norme euclidienne au carré)

\begin{align*}
    \Rightarrow \frac{1}{2}\deriv{ }{t}\mathcal{N}(t) & =(\sigma + \rho)x(t)y (t) -\sigma x^2(t) - y^2(t) - \beta z^2(t)\\
    & \le (\sigma + \rho)x(t)y(t) +\min (1,\sigma,\beta)\mathcal{N}(t)\\
    & \le (\frac{\sigma+\rho}{2})(x^2(t) + y^2(t)) +\min (1,\sigma,\beta) \mathcal{N}(t) &&\mathit{(Young)}\footnotemark \\
    & \le (\frac{\sigma+\rho}{2})(x^2(t) + y^2(t) + z^2(t)) +\min (1,\sigma,\beta) \mathcal{N}(t)\\
    & \le \bigg[\frac{\sigma + \rho}{2} + \min (1,\sigma,\beta) \bigg] \mathcal{N}(t)
\end{align*}
\footnotetext{$\forall p,q \in \mathbb{N}\: \text{tels que} \frac{1}{p}+\frac{1}{q}=1 \Rightarrow \forall a,b \in \R \: ab \le \frac{a^p}{p}+\frac{b^q}{q}$}

Posons $ \eta = \sigma + \rho - 2 \min (1,\sigma,\beta))$. On a alors: 
\[
    \forall t \in \R_+, \  \deriv{}{t}\mathcal{N}(t) \le \eta\  \mathcal{N}(t)
\]
D'après le lemme de Grönwall il vient:
\[
    \forall t \in \R_+,\  \mathcal{N}(t) \le \mathcal{N}_0 e^{\eta t},\  \textrm{avec } \mathcal{N}_0 = \mathcal{N}(0)
\]
Donc la norme du vecteur solution n'explose pas en temps fini.\\

%solution C inf
\underline{\textbf{Proposition:}} Les solution de \eqref{Lorenz} sont $C^\infty$\\
Demonstration:
 Par définition de \eqref{Lorenz} on a que $(x',y',z') = \Gamma(x,y,z)$, or par composition $\Gamma(x,y,z)$ est $C^1$ donc $(x',y',z')$ l'est aussi, ainsi $(x,y,z)$ est $C^2$.De la même manière on obtient par récurence immédiate que $(x,y,z)$ est $C^\infty$\\

%pts stationnaires
On cherche maintenant les points stationnaire de \eqref{Lorenz}.\\
On remarque que $(0,0,0)$ est un point stationnaire, en effet $\Gamma(0,0,0) = 0_{\R^3} \equiv 0$ donc ($\R_+$,0) est une solution de l'equation differentielle.\\
On resout alors $\Gamma(x,y,z)=0$ en supposant que $(x,y,z) \neq 0$, il vient:
\[
\left\{\begin{array}{rl} %O of Gamma
     \sigma(y-x)&=0  \\
     \rho x -y -xz&=0\\
     xy - \beta z&=0
\end{array}\right.
\begin{array}{c} %Num eq
    (L_1)\\
    (L_2)\\
    (L_3)
\end{array}
\]
de $(L_1)$ on obtient que $x=y$. Dans $(L_2)$ et dans $(L_3)$ on remplace $y$ par $x$, il vient alors:
\begin{gather*}
    (L_2) \Rightarrow \rho x - x - xz = 0 \Rightarrow x (\rho -1 -z ) = 0 \\
    (L_3) \Rightarrow x^2 - \beta z = 0 \Rightarrow z = \frac{x^2}{\beta}
\end{gather*}
On obtient ainsi:
\begin{align*}
    (L_2) & \Rightarrow x (\rho - 1 - \frac{x^2}{\beta}) = 0 \text{, or }x \neq 0\\
        & \Rightarrow x^2 = \beta (1-\rho)\\
    \text{Si } \beta(1-\rho) \ge 0 & \Leftrightarrow \beta \ge 0,\rho\le 1 \text{ ou } \beta \le 0,\rho\ge 1\text{ alors:}\\
    &\Rightarrow x = \sqrt{\beta(1-\beta)}
\end{align*}

De ces trois équation on obtient que:
\[
    \Gamma(x,y,z)=0_{\R^3} \Rightarrow (x,y,z) = (\pm \sqrt{ \beta (\rho -1)} ,\pm \sqrt{\beta (\rho -1)}, \rho -1)
\]

On verifie aisément que cette relation est une \'equivalance, en ramplacant les valeurs obtenue de $x$,$y$ et $z$ dans $\Gamma(x,y,z)$
\\
\underline{Remarque:} Si $\rho = 1$ alors il n'y a qu'un seul équilibre

On se propose d'étudier la stabilité des points stationnaires. Pour cela on s'intéresse à la linéarisé de \eqref{Lorenz}, donné par:
\[
\colv{x'\\y'\\z'} = \mathcal{D}_{\Gamma}(x_s,y_s,z_s)\colv{x\\y\\z}    
\]
avec $(x_s,y_s,z_s)$ les coordonnées des points stationnaire, $\mathcal{D}_{\Gamma}(x,y,z)$ la differentielle de $\Gamma$ donné par:
\[
\mathcal{D}_{\Gamma}(x,y,z) =
\begin{pmatrix}
    \sigma & \sigma & 0 \\ \rho - z & -1 & -x \\ y & x & - \beta
\end{pmatrix}
\]
On étudie premièrement l'équilibre autour de $0_{\R^3}$:\\
L'équation ainsi obtnue est:
\begin{equation}
    \colv{x'\\y'\\z'} =
    \begin{pmatrix}
        - \sigma & \sigma & 0 \\ \rho & -1 & 0 \\ 0 & 0 & - \beta
    \end{pmatrix}
    \colv{x \\ y \\ z}
\end{equation}
Autrement dit on obtient:
\begin{equation}
    \left\{\begin{array}{lr}
        x' = \sigma (y-x) \\
        y' = \rho x - y \\
        z' = \beta z
    \end{array}\right.
\end{equation}


%caractérisation de l'équilibre 0
On se propose de caractériser l'équilibre $0_{\R^3}$.\\
\textit{Rappel:} Théoreme:\\
Soient $f\in C^2(U;E)$ et $v\in U$ tel que $f(v)=0$. Si $\max\{\Re(\lambda); \lambda\in \mathrm{Sp}(\mathcal{D}_f(v))\}$ est atteint pour une valeur propre de $\mathcal{D}_f(v)$ de partie réelle strictement positive. Alors $v$ est un point d'équilibre instable pour l'équation $u'=f(u)$ \\
Calculons le polynome caractéristique de la differentielle de $\Gamma$ en $0_{\R^3}$, on notera se polynome $\chi$.
\[
    \chi (\lambda) = \det\big(\lambda\id - \mathcal{D}_{\Gamma}(0,0,0)\big) = (\lambda - \beta)(\lambda^2 + \lambda(\sigma+1)+\sigma-\rho\sigma)
\]
Remarque: $\beta$ est toujours racine de $\chi$\\
Posons $P:\lambda \in \R \mapsto \lambda^2 + \lambda(\sigma+1)+\sigma-\rho\sigma,\text{ donc }\chi(\lambda)=(\lambda-\beta)P(\lambda)$, on calcule le deteterminant de $P$:
\[
  \Delta = (\sigma+1)^2 - 4(\sigma-\sigma\rho)) = (\sigma-1)^2 +4\sigma\rho
\]
On obtient alors un autre polynome. On étudie son signe et on différencie les cas suivant:

\underline{Cas 1: Si $(\rho,\sigma)=(0,1)$:}\\
Alors $\chi_A(x) = (\lambda-\beta)(\lambda+1)^2$, donc d'après le théoreme, si $\beta$ est strictement positif $0$ est un point d'équilibre instable pour \eqref{Lorenz}

\underline{Cas 2: Si $(\rho,\sigma)=(1,-1)$:}\\
Alors $\chi_A(\lambda) = (\lambda-\beta)\lambda^2$, donc d'après le théoreme, si $\beta$ est strictement positif (resp. strictement négatif) $0_{\R^3}$ est un point d'équilibre instable (resp. asymptotiquement stable) pour \eqref{Lorenz}


\underline{Cas 3: Si $\rho \in ]0,1[, \sigma \in \mathbb{R}$:}\\
$\Delta> (\sigma-1)^2 >0$ donc $P$ a deux racines réelles.
que l'on note: 
\[
    \lambda_\pm = \frac{-(\sigma-1)\pm \sqrt{(\sigma-1)^2 + 4\sigma\rho}}{2}
\]
Regardons le signe des racines. $\lambda_+ < -(\sigma-1)+ \sqrt{(\sigma+1)^2} \le 0 $, de même pour $\lambda_-$, on a, $\lambda_- < - \sigma +1- |\sigma - 1| \le 0$. De plus si $\beta$ est strictement négatif, toutes les racines sont strictement négative donc $0_{\R^3}$ est un point d'équilibre asymptotiquement stable pour \eqref{Lorenz}. Si $\beta > 0$, $0_{\R^3}$ est un point d'équilbre instable.

\underline{Cas 4: Si $\rho > 1$ et $\sigma \ge 0 $: }\\
Dans ce cas on a $\Delta > 0$, on retrouve la même expression des racines que précédemment $\lambda_\pm$. On trouve que $2\lambda_+ > 1-\sigma+|\sigma+1| \ge 0$, donc $\lambda_+ > 0$, pour $\lambda_-$ on a, $2\lambda_- < 1-\sigma-|\sigma+1| \le 0 $, donc $\lambda_- < 0$, dans ce cas $0_{\R^3}$ est un équilibre instable pour \eqref{Lorenz}.

\underline{Cas 5: Si $\rho<0$ et $\sigma\le 0$:}\\
Comme dans le cas précédent on trouve $\Delta>0$ avce cette fois $\lambda_+ < 0$ et $\lambda_- >0$. On le retrouve en majorant $2\lambda_+$ et $2\lambda_-$ par $\rho=0$, on majore ainsi $\lambda_+$ et minore $\lambda
_-$. Donc $0_{\R^3}$ est un équilbre instable pour \eqref{Lorenz}.


\underline{Remarque:} Les cas 1,3,4,5 sont des point d'équilibre hyperbolique de \eqref{Lorenz} en effet, $\mathrm{Sp}(A)\cap \mathrm{Vect}(i)= \emptyset$



\end{document}