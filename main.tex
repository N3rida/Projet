\documentclass{article}
\usepackage[french]{babel}
\usepackage[T1]{fontenc}

\usepackage{graphicx} % Required for inserting images
\usepackage{amsmath}
\usepackage{amsfonts}
\usepackage{tikz}

\title{Système de Lorenz}
\author{Salomé COUDIERE, Louka OUKALA, Adrien CHABERT}
\date{Semestre 6}

\newcommand*\colv[1]{
\left(\begin{array}{c}
    #1
\end{array}\right)
}
\newcommand{\R}{\mathbb{R}}

\newcommand{\deriv}[3][ ]{
    \ensuremath{\frac{\mathrm{d}^{#1}#2}{\mathrm{d}^{#1} #3}}
}

\newcommand{\cad}{c'est-\`a-dire }

\begin{document}

\begin{equation}
    \label{Lorenz}
    \left\{
    \begin{array}{rl}
        x' &=\sigma(y-x) \\
        y' &=\rho x -y - xz\\
        z' &=xy - \beta z
    \end{array}
    \right.
    \begin{array}{r}
        (L_1)\\
        (L_2)\\
        (L_3)
    \end{array}
\end{equation}
On peut récrire ce systeme de la manière suivante.
\begin{equation}
\deriv{\Vec{u}}{t} = \Gamma(\Vec{u}), \quad
\Gamma : \Vec{u} = \colv{x\\y\\z} \in \R^3 \mapsto \colv{\sigma(y-x) \\ \rho x-y-xz \\ xy-\beta z}
\end{equation}
Remarquons que $\Gamma$ est polynomiale, donc $\Gamma$ est de classe $C^\infty$ en particulier elle est $C^1$. 
D'après le théoreme de Cauchy-Lipschitz, l'équation \eqref{Lorenz} admet une unique solution maximale de classe $C^1$ que l'on notera $(I,(x,y,z))$ avec $I \subset \R_+$ avec $I$ de la forme $]0,T[,\ T \in ]0,+\infty]$. Montrons que $(I,(x,y,z))$ est globale. Dans \eqref{Lorenz} on s'intéresse à la quantité: $x(L_1) + y(L_2) + z(L_3)$
\[
    xx'+yy'+zz' = \sigma yx - \sigma x^2 + \rho xy - y^2 -xyz + xyz - \beta z^2\\
\]  
Posons $\mathcal{N}: (x,y,z) \in \R^3 \mapsto x^2 + y^2 + z^2$ (N est la norme euclidienne au carré)
\begin{align*}
    \Rightarrow \frac{1}{2}\deriv{ }{t}\bigg(\mathcal{N}(x,y,z)\bigg) & = & (\sigma + \rho)xy -\sigma x^2 - y^2 - \beta z^2\\
    & \le & (\sigma + \rho)xy -\min (1,\sigma,\beta)\mathcal{N}(x,y,z)\\
    \mathit{(Young)}\footnotemark & \le & (\frac{\sigma+\rho}{2})(x^2 + y^2) -\min (1,\sigma,\beta) \mathcal{N}(x,y,z)\\
    & \le & (\frac{\sigma+\rho}{2})(x^2 + y^2 + z^2) -\min (1,\sigma,\beta) \mathcal{N}(x,y,z)\\
    & \le & \bigg[\frac{\sigma + \rho}{2} - \min (1,\sigma,\beta) \bigg] \mathcal{N}(x,y,z)
\end{align*}
\footnotetext{$\forall p,q \in \mathbb{N}\: \text{tels que} \frac{1}{p}+\frac{1}{q}=1 \Rightarrow \forall a,b \in \R \: ab \le \frac{a^p}{p}+\frac{b^q}{q}$}

Posons $ \eta = \sigma + \rho - 2 \min (1,\sigma,\beta))$. On a alors: 
\[
    \forall t \in \R_+, \: \deriv{}{t}\bigg( \mathcal{N}(x,y,z)\bigg) \le \eta\: \mathcal{N}(x,y,z)
\]
D'après le lemme de Grönwall il vient:
\[
    \forall t \in \R_+,\: \mathcal{N}(x,y,z)(t) \le \mathcal{N}_0 e^{\eta t},\: \textrm{avec } \mathcal{N}_0 = \mathcal{N}(x,y,z)(0)
\]
Donc la norme du vecteur solution n'explose pas en temps fini. En effet supposons par l'absurde que la norme du vecteur (x,y,z) explose en temps fini \cad: 
\[
\exists t_0 \in \R_+ \textrm{tel que } \lim_{t \to t_0} \mathcal{N}(x,y,z)(t)=+\infty \  \textrm{mais, } \\
\lim_{t\to t_0} \mathcal{N}_0 e^{\eta t} = \mathcal{N}_0 e^{\eta t_0} \le + \infty
\]
Or,
\[\forall t \in \R_+ \: \mathcal{N}(x,y,z)(t) \le \mathcal{N}_0 e^{\eta t}
\]On obtient alors une absurdité. Donc $\mathcal{N}(x,y,z)$ n'explose pas en temps fini. On en déduit que $(x,y,z)$ n'explose pas en temps fini. En effet suposons que une des composante explose en temps fini par exemple $x$, \cad:
\begin{gather*}
    \exists t_* \in \R_+ \: \textrm{tel que} \lim_{t \to t_*} x(t) = + \infty\\
    \textrm{Or, } \forall t \in \R_+ \: x(t) \le x^2(t) + y^2(t) + z^2(t) \le \mathcal{N}_0 e^{\eta t}
\end{gather*}
En passant \`a la limte dans l'inéquation précédente on obtient:
\[
    lim_{t \to t_*} x(t) < \lim_{t \to t_*} \mathcal{N}_0 e^{\eta t}
\]
c'est une absurdité donc $(x,y,z)$ est une solution globale de \eqref{Lorenz} \cad $I=\R_+$

%solution C inf
Par définition de \eqref{Lorenz} on a que $(x',y',z') = \Gamma(x,y,z)$, or par composition $\Gamma(x,y,z)$ est $C^1$ donc $(x',y',z')$ l'est aussi ainsi $(x,y,z)$ est $C^2$.De la même manière on obtient par récurence immédiate que $(x,y,z)$ est $C^\infty$\\
%pts stationnaires
On cherche maintenant les points stationnaire de \eqref{Lorenz}.\\
On remarque que $(0,0,0)$ est un point stationnaire, en effet $\Gamma(0,0,0) = 0_{\R^3} \equiv 0$ donc ($\R$,0) est une solution de l'equation differentielle.\\
On resout alors $\Gamma(x,y,z)=0$ en supposant que $(x,y,z) \neq 0$, il vient:
\[
\left\{\begin{array}{rl} %O of Gamma
     \sigma(y-x)&=0  \\
     \rho x -y -xz&=0\\
     xy - \beta z&=0
\end{array}\right.
\begin{array}{c} %Num eq
    (L_1)\\
    (L_2)\\
    (L_3)
\end{array}
\]
de $(L_1)$ on obtient que $x=y$. Dans $(L_2)$ on remplace $y$ par $x$, il vient alors:
\[
    (L_2) \Rightarrow \rho x - x - xz = 0 \Rightarrow x (\rho -1 -z ) = 0 \Rightarrow z = \rho -1
\]
De m\^eme dans $(L_3)$
\[
    (L_3) \Rightarrow x^2 - \beta z = 0 \Rightarrow z = \pm \sqrt{\beta z}
\]
De ces trois équation on obtient que:\[
    \Gamma(x,y,z)=0 \Rightarrow (x,y,z) = (\pm \sqrt{ \beta (\rho -1)} ,\pm \sqrt{\beta (\rho -1)}, \rho -1)   
\]
On verifie aisaiment que cette relation est une \'equivalance, en ramplacant les valeurs obtenue de $x$,$y$ et $z$ dans $\Gamma(x,y,z)$\\

Dans la partie précédente on a obtenue l'inégalité suivante:\[
    \forall t \in \R_+,\: \mathcal{N}(x,y,z)(t) \le \mathcal{N}_0 e^{\eta t},\: \textrm{avec } \mathcal{N}_0 = \mathcal{N}(x,y,z)(0)
\]
On remarque ainsi que selon le signe de $\eta$ notre solution $(\R_+,(x,y,z))$ peut être bornée, dans le cas de notre étude $\sigma = 10,\ \rho = 28,\ \beta = \frac{8}{3}$ on a $\eta = 10 + 28 - 2*1 = 36$


\end{document}
