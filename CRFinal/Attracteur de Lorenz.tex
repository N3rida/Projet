\documentclass{article}
\usepackage[french]{babel}
\usepackage[T1]{fontenc}

\usepackage{graphicx} % Required for inserting images
\graphicspath{{C:/Users/chabe/Documents/L3/PROJET/Projet/CRFinal/ressources/}}
\usepackage{amsmath}
\usepackage{amsfonts}
\usepackage{tikz}
\usepackage{fancyhdr}
\usepackage{thmbox}

\pagestyle{fancy}
\fancyhead{}
\fancyfoot{}
\fancyhead[L]{Attracteur de Lorenz - \thesection}
\fancyfoot[C]{\thepage}

\title{Système de Lorenz}


\newcommand*\colv[1]{
\left(\begin{array}{c}
    #1
\end{array}\right)
}

\newcommand{\R}{\mathbb{R}}

\newcommand{\deriv}[3][ ]{
    \ensuremath{\frac{\mathrm{d}^{#1}#2}{\mathrm{d}^{#1} #3}}
}
\newcommand{\id}[1][]{\ensuremath{\mathrm{Id}_{#1}}}

\newcommand{\cad}{c'est-\`a-dire }

\newtheorem[M , nocut]{prop}{Proposition}[section]
\newtheorem[M]{propt}{Propriété}[section]
\newtheorem[L , nocut]{thm}{Théoreme}
\newtheorem[L]{cor}{Corollaire}

\begin{document}
\begin{titlepage}
    \vfil
    %\includegraphics{LorenzTransparent}
\end{titlepage}
\section*{Introduction}

\begin{equation}
    \label{Lorenz}
    \left\{
    \begin{array}{rl}
        x' &=\sigma(y-x) \\
        y' &=\rho x -y - xz\\
        z' &=xy - \beta z
    \end{array}
    \right.
    \begin{array}{r}
        (L_1)\\
        (L_2)\\
        (L_3)
    \end{array}
\end{equation}

\begin{equation}
    \label{fLorenz}
    \deriv{\Vec{u}}{t} = \Gamma(\Vec{u}), \quad
    \Gamma : \Vec{u} = \colv{x\\y\\z} \in \R^3 \mapsto \colv{\sigma(y-x) \\ \rho x-y-xz \\ xy-\beta z}
    \end{equation}

\section{Méthodes numériques}


\section{Existence et première propriété}

Dans cette section nous allons faire une étude du systeme de Lorenz afin de determiner l'existence des solutions. Nous nous limiterons à l'étude des temps positifs ($t \ge 0$)

\begin{prop}
    Le systeme de Lorenz admet des solutions. De plus les solutions sont globales sur $\R_+$ et de classe $C^1$
\end{prop}

\begin{proof}
    $\Gamma$ est $C^1$ donc elle est localement Lipschitzienne, de plus elle de depend pas directement du temps. D'après le théoreme de Cauchy-Lipschitz, l'équation \eqref{Lorenz} admet une unique solution maximale de classe $C^1$ que l'on notera $(I,(x,y,z))$ avec $I \subset \R_+$ avec $I = [0,T[,\ T \in ]0,+\infty]$. Montrons que $(I,(x,y,z))$ est globale. Dans \eqref{Lorenz} on s'intéresse à la somme de, $x$ fois première ligne avec $y$ fois la seconde ligne et $z$ fois la troisième ligne.
\[
    xx'+yy'+zz' = \sigma yx - \sigma x^2 + \rho xy - y^2 -xyz + xyz - \beta z^2\\
\]  
Posons $\mathcal{N}: (t) \in \R^3 \mapsto x(t)^2 + y(t)^2 + z(t)^2$ ($\mathcal{N}$ est la norme euclidienne au carré)

\begin{align*}
    \Rightarrow \frac{1}{2}\deriv{ }{t}\mathcal{N}(t) & =(\sigma + \rho)x(t)y (t) -\sigma x^2(t) - y^2(t) - \beta z^2(t)\\
    & \le (\sigma + \rho)x(t)y(t) +\min (1,\sigma,\beta)\mathcal{N}(t)\\
    & \le (\frac{\sigma+\rho}{2})(x^2(t) + y^2(t)) +\min (1,\sigma,\beta) \mathcal{N}(t) &&\mathit{(Young)}\footnotemark \\
    & \le (\frac{\sigma+\rho}{2})(x^2(t) + y^2(t) + z^2(t)) +\min (1,\sigma,\beta) \mathcal{N}(t)\\
    & \le \bigg[\frac{\sigma + \rho}{2} + \min (1,\sigma,\beta) \bigg] \mathcal{N}(t)
\end{align*}
\footnotetext{$\forall p,q \in \mathbb{N}\: \text{tels que} \frac{1}{p}+\frac{1}{q}=1 \Rightarrow \forall a,b \in \R \: ab \le \frac{a^p}{p}+\frac{b^q}{q}$}

Posons $ \eta = \sigma + \rho - 2 \min (1,\sigma,\beta)$. On a alors: 
\[
    \forall t \in \R_+, \  \deriv{}{t}\mathcal{N}(t) \le \eta\  \mathcal{N}(t)
\]
D'après le lemme de Grönwall il vient:
\[
    \forall t \in \R_+,\  \mathcal{N}(t) \le \mathcal{N}_0 e^{\eta t},\  \textrm{avec } \mathcal{N}_0 = \mathcal{N}(0)
\]
Donc la norme du vecteur solution n'explose pas en temps fini.
\end{proof}

En ayant obtenue les résultats de la proposition on peut aisément en déduire que la propriété suivante.

\begin{propt}
    Les solutions du système de Lorenz \eqref{Lorenz} sont de classe $C^\infty$
\end{propt}

\begin{proof}
    Par \eqref{fLorenz} on a que $(x',y',z') = \Gamma(x,y,z)$, or par composition $\Gamma(x,y,z)$ est $C^1$ donc $(x',y',z')$ l'est aussi, ainsi $(x,y,z)$ est $C^2$.De la même manière on obtient par récurence immédiate que $(x,y,z)$ est $C^\infty$
\end{proof}

Cette propriété nous permet de dire que même si \eqref{Lorenz} est un système dit chaotique, on peut affirmer que les solutions sont régulière.

\section{\'Etude des points stationnaires}
Après une première etude sommaire du système de Lorenz \eqref{Lorenz}, on s'intéresse maintenant à l'étude des points stationnaires. Les points stationnaires sont des points tels que si le système passe par un de ces points il y reste indéfiniment.

%vérifier la formulation de ce théoreme
\begin{thm}
    Soit l'équation differentielle bien définie,
    \begin{equation*}
        u'=f(u)
    \end{equation*}
    Si il existe $v$ tel que $f(v)=0$ alors $v$ est un point stationnaire de l'équation differentielle.
\end{thm}

\subsection{Rappel des théorèmes}
\label{sec:Rappel-des-théorèmes}
D'abord on rappelle les résultats sur lesquels on s'appuiras dans la suite. Ces théorèmes établissent des résultats de stabilité sur les point stationnaires en se basant sur l'étude spectrale de la différentielle.
\begin{thm}
    \label{thm:eq-ass-stable}
    Soit $f\in C^2(U;E)$ sur un ouvert $U$ d'un espace de Banach $E$ et $v\in U$ tel que $f(v)=0$. Si le spectre de $\mathcal{D}_\Gamma(v)$ est inclus dans le demi-plan ouvert $\left\{\lambda; \Re(\lambda)<0\right\}$ alors $v$ est un point d'équilibre assymptotiquement stable pour l'équation $u'=f(u)$
\end{thm}

Ce résultat nous permet de conclure sur le comportement assymtotique des solutions ayant subit une faible perturbation à l'instant de départ (\cad les solution $(\R_+,u_1)$ telles que $u_1(t_{\text{init}})=u(t_{\text{init}})+\varepsilon,\ 0<\|\varepsilon\|\ll 1 $ ). On obtient ainsi $\lim_{t\to\infty}\|u-u_1\| = 0$.

%resultat pour la stabilité simple ?

\begin{thm}
    \label{thm:eq-instable}
    Soit $f\in C^2(U;E)$ et $v\in U$ tel que $f(v)=0$. Si $\max\{\Re(\lambda); \lambda\in \mathrm{Sp}(\mathcal{D}_f(v))\}$ est atteint pour une valeur propre de $\mathcal{D}_f(v)$ de partie réelle strictement positive. Alors $v$ est un point d'équilibre instable pour l'équation $u'=f(u)$
\end{thm}

% commentaire sur le résultat

\subsection{Détermination des équilibres}
Dans un premier temps on regarde si notre système possèdes des équilibres et si oui lesquels.
\begin{prop}
    L'equation \eqref{Lorenz} possèdes 3 points d'équilibre qui sont:
    \begin{align*}
        0_{\R^3} &&   S_+ :=\colv{\sqrt{ \beta (\rho -1)} \\ \sqrt{\beta (\rho -1)}\\ \rho -1}  &&  S_- := \colv{-\sqrt{ \beta (\rho -1)} \\ - \sqrt{\beta (\rho -1)}\\ \rho -1}
    \end{align*}
\end{prop}

\begin{proof}
On remarque que $(0,0,0)$ est un point stationnaire, en effet $\Gamma(0,0,0) = 0_{\R^3}:= 0$ donc ($\R_+$,0) est une solution de l'equation differentielle.\\
On resout alors $\Gamma(x,y,z)=0$ en supposant que $(x,y,z) \neq 0$, il vient:
\[
\left\{\begin{array}{rl} %O of Gamma
     \sigma(y-x)&=0  \\
     \rho x -y -xz&=0\\
     xy - \beta z&=0
\end{array}\right.
\begin{array}{c} %Num eq
    (L_1)\\
    (L_2)\\
    (L_3)
\end{array}
\]
de $(L_1)$ on obtient que $x=y$. Dans $(L_2)$ et dans $(L_3)$ on remplace $y$ par $x$, il vient alors:
\begin{gather*}
    (L_2) \Rightarrow \rho x - x - xz = 0 \Rightarrow x (\rho -1 -z ) = 0 \\
    (L_3) \Rightarrow x^2 - \beta z = 0 \Rightarrow z = \frac{x^2}{\beta}
\end{gather*}
On obtient ainsi un polynome de degrès 3 il y a donc au plus 3 équilibres:
\begin{align*}
    (L_2) & \Rightarrow x (\rho - 1 - \frac{x^2}{\beta}) = 0 \text{, or }x \neq 0\\
        & \Rightarrow x^2 = \beta (1-\rho)\\
    \text{Si } \beta(1-\rho) \ge 0 & \Leftrightarrow \beta \ge 0,\rho\le 1 \text{ ou } \beta \le 0,\rho\ge 1\text{ alors:}\\
    &\Rightarrow x = \pm \sqrt{\beta(1-\beta)}
\end{align*}

De ces trois équation on obtient que:
\[
    \Gamma(x,y,z)=0_{\R^3} \Rightarrow (x,y,z) = (\pm \sqrt{ \beta (\rho -1)} ,\pm \sqrt{\beta (\rho -1)}, \rho -1)
\]

On verifie aisément que cette relation est une \'equivalance, en ramplacant les valeurs obtenue de $x$,$y$ et $z$ dans $\Gamma(x,y,z)$
\end{proof}

\begin{example}[Remarque]
    Si $\rho=1$ il n'y a qu'un seul équilibre.
\end{example}

\subsection{Caractérisation de l'équilibre $0_{\R^3}$}
Afin d'utiliser les outils introduit dans la partie \ref{sec:Rappel-des-théorèmes}, on calcule le polynome caractéristique de la differentielle en $0$ (noté $\chi$) donné par le calcul suivant.
\[
    \chi (\lambda) = \det\big(\lambda\id - \mathcal{D}_{\Gamma}(0,0,0)\big) = (\lambda - \beta)(\lambda^2 + \lambda(\sigma+1)+\sigma-\rho\sigma)
\]
\begin{example}[Remarque]
    $\beta$ est toujours racine de $\chi$
\end{example}
On remarque que $\chi$ est un polynome de degrè 3 et d'après la remarque précédente on a que $\chi(\lambda) = (\lambda - \beta)P(\lambda)$, avec $P:\lambda \in \R \mapsto \lambda^2 + \lambda(\sigma+1)+\sigma-\rho\sigma$. Ainsi les pour étudier $\chi$, on calcule le deteterminant de $P$:
\[
  \Delta = (\sigma+1)^2 - 4(\sigma-\sigma\rho) = (\sigma-1)^2 +4\sigma\rho
\]

Dans la suite on vas dissocier les cas suivant:
\begin{enumerate}
    \item Cas $\Delta > 0$:
    \item Cas $\Delta = 0$:
    \item Cas $\Delta < 0$:
\end{enumerate}

\subsubsection*{Cas $\Delta > 0$}
Premièrement on détermine les conditions pour que $\Delta$ soit strictement positif.
\begin{prop}
    $\Delta$ est strictement positif pour l'ensemble des paramètre suivant:
    \[
     \left\{(\sigma,\rho)\in \R ^2 : 1-2 \rho - \sqrt{\frac{(4\rho-2)^2}{4} -1 } > \sigma\ et \ \sigma > 1-2 \rho + \sqrt{ \frac{(4\rho-2)^2}{4} -1  } \right\}    
    \]
\end{prop}
\begin{example}
    Dans le cas $\Delta > 0$, on a $\rho \in ]0,1[$
\end{example}
\begin{proof}
    $\Delta = \sigma^2 + \sigma(4\rho-2) + 1$ est un polynome de la variable $\sigma$. On note le deteterminant de $\Delta$ selon la variable $\sigma$, $\delta$. $\Delta$ est convexe par rapport à la variable $\sigma$ donc pour que $\Delta$ soit strictement positif il faut que $\delta$ soit strictement négatif. On cherche alors $\rho$ tel que \[ \delta<0\ \Leftrightarrow\ 16\rho^2 - 16\rho <0\ \Leftrightarrow\ \rho(\rho-1)<0\ \Leftrightarrow \rho \in ]0,1[ \]
    %finir demo !
\end{proof}
On s'intéresse a la stabilité des points stationnaires dans le cas où $\Delta>0$
%A faire ...

\subsubsection*{Cas $\Delta = 0$}
On s'intéresse maintenant aux cas tels que $\Delta=0$

\begin{prop} 
    \label{prop:Deg0}
    $\Delta$ est nul si et seulemment si on a la paramétrisation suivante,
    \[
        \left\{(\sigma,\rho)\in \R ^2 :\sigma = 1-2 \rho + \sqrt{ \frac{(4\rho-2)^2}{4} -1 }\ et\ \sigma = 1-2 \rho - \sqrt{ \frac{(4\rho-2)^2}{4} -1 } \right\}  
    \]
\end{prop}

\begin{example}[Remarque]
    Cette paramétrisation implique que $\rho \in \R \setminus ]0,1[$ 
\end{example}

\begin{proof}
    A nouveaux on considère $\Delta$ comme un polynome de la variable $\sigma$. Pour que $\Delta$ ait des racines il faut que $\delta$ soit positif ou nul.
    \[
    \delta \ge 0 \Leftrightarrow \rho(\rho-1) \ge 0 \Leftrightarrow \rho \in \R \setminus ]0,1[
    \] Maintenant on calcule les racines de $\Delta$.
    \begin{align*}
        \sigma &= \frac{2-4\rho + \sqrt{ (4\rho-2)^2 -4 }}{2} = 1-2 \rho + \sqrt{ \frac{(4\rho-2)^2}{4} -1 }\\
        \sigma &= \frac{2-4\rho - \sqrt{ (4\rho-2)^2 -4 }}{2} = 1-2 \rho - \sqrt{ \frac{(4\rho-2)^2}{4} -1 }
    \end{align*}
    On obtient ainsi la paramétrisation voulue.
\end{proof}
Maintenant on s'intéresse à la stabilité de l'équilibre pour le cas $\Delta=0$.
\begin{prop}\label{prop:eqDeg0}
    Dans le cas $\Delta=0$:
    Si $\beta >0$ l'équilbre est instable si $\beta<0$ alors on a:
    \begin{itemize}
        \item Si $\sigma < -1 , \rho > 1$ alors $0_{\R^3}$ est un equilibre instable pour \eqref{Lorenz}
        \item Si $\sigma > -1 , \rho < 1$ alors $0_{\R^3}$ est un equilibre assymptotiquement stable pour \eqref{Lorenz}
    \end{itemize}
\end{prop}
\begin{proof}
    Comme $\beta$ est toujours racine si $\beta >0$ alors $\max (\mathrm{Sp}(\mathcal{D}_\Gamma)) \ge \beta > 0$ donc d'après le theoreme \ref{thm:eq-instable} $0_{\R^3}$ est un équilbre instable pour \eqref{Lorenz}\\
    Si $\beta < 0$ alors dans le cas de $\Delta = 0, \mathrm{Sp}(\mathcal{D}_\Gamma) = \{\beta, -\frac{\sigma+1}{2}\}$, on cherche alors les conditions sur les paramètre $\sigma$ et $\rho$ telles que le système soit instable ou assymptotiquement stable.
    On s'intéresse au cas instable et on inversera le sens des inégalités pour avoir le cas assymptotiquement stable. 
    \[
        -\frac{\sigma+1}{2} >0 \Leftrightarrow \sigma < -1   
    \]% modifier pour integrer la ref a la parametrisation
    Pour "inserer ref param" on obtient que $-1 > 1-2 \rho + \sqrt{ \frac{(4\rho-2)^2}{4} -1 }$ n'as pas de solutions.
    Pour "inserer ref param" cherche $\rho$ tel que:
    \[
        1-2 \rho - \sqrt{ \frac{(4\rho-2)^2}{4} -1 } < -1 \Rightarrow \frac{(4\rho-2)^2}{4} -1 > (2\rho-2)^2 \Leftrightarrow \rho > 1 
    \]En faisant les caluls avec l'inégalités inverse on retrouve le cas assymptotiquement stable.
\end{proof}

\subsubsection*{Cas $\Delta < 0$}
On determine les cas tels que $\Delta<0$
\begin{prop}
    $\Delta$ est strictement négatif pour l'ensemble des paramètre suivant:
    \[
     \left\{(\sigma,\rho)\in \R ^2 : 1-2 \rho - \sqrt{ \frac{(4\rho-2)^2}{4} -1 }< \sigma < 1-2 \rho + \sqrt{ \frac{(4\rho-2)^2}{4} -1 } \right\}    
    \]
\end{prop}
\begin{example}[Remarque]
    Dans cet ensemble on a que: $\rho \in \R \setminus [0,1]$ 
\end{example}

\begin{proof}
    Pour que $\Delta$ ait des racines il faut que $\delta$ soit strictement positif.
    \[
    \delta > 0 \Leftrightarrow \rho(\rho-1) > 0 \Leftrightarrow \rho \in \R \setminus [0,1]
    \]Par convexité de $\Delta$ on a que $\Delta$ est négatif entre ces racines, il vient alors:
    \[
        1-2 \rho - \sqrt{ \frac{(4\rho-2)^2}{4} -1 } < \sigma < 1-2 \rho - \sqrt{ \frac{(4\rho-2)^2}{4} -1 }
    \]Donc l'ensemble des $\rho,\sigma$ qui verifie cette inégalité sont alors des paramètres tels que $\Delta$ est négatif.
\end{proof}
On determine les cas pour lesquels l'equilibre est instable ou assymptotiquement stable.
\begin{prop}
    Dans le cas $\Delta<0$:
    Si $\beta >0$ l'équilbre est instable si $\beta<0$ alors on a:
    \begin{itemize}
        \item Si $\sigma < -1 , \rho > 1$ alors $0_{\R^3}$ est un equilibre instable pour \eqref{Lorenz}
        \item Si $\sigma > -1 , \rho < 1$ alors $0_{\R^3}$ est un equilibre assymptotiquement stable pour \eqref{Lorenz}
    \end{itemize}
\end{prop}
\begin{proof}
    Pour la différenciation du cas de $\beta$ cf. la démonstration de \ref{prop:eqDeg0}. Dans ce cas on a:
    \[
        \mathrm{Sp}(\mathcal{D}_\Gamma) = \left\{\beta, \omega := \frac{-(\sigma+1)+ i \sqrt{-\Delta}}{2}, \bar{\omega} := \frac{-(\sigma+1)- i \sqrt{-\Delta}}{2}\right\}
    \]Pour étudier la stabilité on s'intéresse à la partie réelle de $\omega$ et $\bar{\omega}$. Or $\Re (\omega) = \Re (\bar{\omega}) = -\frac{\sigma+1}{2}$
\end{proof}
\subsection{Caractérisation de $S_\pm$}
\section{Annexes}


\end{document}